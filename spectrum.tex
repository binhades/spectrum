%%
%% Beginning of file 'spectrum.tex'
%%
%% using aastex version 6.1
\documentclass[manuscript]{aastex61}
\hypersetup{linkcolor=red,citecolor=green,filecolor=cyan,urlcolor=magenta}
%\shorttitle{Spectral line average}
%\shortauthors{B. Liu et al.}
\watermark{DRAFT}
\begin{document}

\title{On the method of averaging spectra from an extended source }


%\author{}
%\affiliation{}
%\affiliation{}


\begin{abstract}
To get the spectral feature of the extended astronomical sources, it is not clear how many pixels of the source region should be summed.

\end{abstract}

\keywords{spectrum}

\section{Introduction} \label{sec:intro}
Three dimentional data cubes are common format for spectropy mapping products.

\begin{enumerate}
\item 
\item 
\item 
\item 
\item 
\item 
\end{enumerate}

\section{develop the spectra average method} \label{sec:method}
\section{Results} \label{sec:result}
\section{Discussion} \label{sec:disc}

\acknowledgments

\appendix


%\begin{thebibliography}{}
%\end{thebibliography}

\end{document}

